\documentclass[../doc.tex]{subfiles}
\begin{document}

\section{Integrate and Fire Model}

\begin{equation}
CV = Q 
\end{equation}

\begin{equation}\label{eq:current}
C \frac{dV}{dt} = I 
\end{equation}

Recall Ohm's Law:

\begin{equation}
IR = V \Rightarrow I = \frac{V}{R} 
\end{equation}

At rest, the neuron is actually subject to discharge, hence the minus sign:

\begin{equation}
I = \frac{-V}{R} 
\end{equation}

But this is offset by the ATP pump:

\begin{equation}
I = I_{pump} - \frac{V}{R} 
\end{equation}

Where $I_{pump}$ is the polarizing current at rest: $\frac{V_{rest}}{R}$. So:

\begin{equation}
I = I_{pump} - \frac{V}{R} = \frac{V_{rest}}{R} - \frac{V}{R} = \frac{1}{R} (V_{rest} - V)
\end{equation} 

Now let's add an external current, $I_ext$:

\begin{equation}
I = frac{1}{R} (V_{rest} - V) + I_ext
\end{equation} 

From ~\ref{eq:current}:

\begin{equation}
C \frac{dV}{dt} = I = frac{1}{R} (V_{rest} - V) + I_ext
\end{equation} 

Multiply both sides by $R$:

\begin{equation}\label{eq:RC}
RC \frac{dV}{dt} = V_{rest} - V + RI_ext
\end{equation} 

Define the membrane time constant $\tau$:

\begin{equation}
RC \equiv \tau
\end{equation} 

Plug $\tau$ into equation ~\ref{eq:RC}:

\begin{equation}\label{eq:tau}
\tau \frac{dV}{dt} = V_{rest} - V + RI_{ext}
\end{equation} 

In any given instant? $V_{rest}$ and $RI_{ext}$ are constants. IS THIS A DEFINITION:

\begin{equation}\label{eq:infty}
V_{\infty} = V_{rest} + RI_{ext}
\end{equation} 

So, substituting ~\ref{eq:infty} into ~\ref{eq:tau}:

\begin{equation}
\tau \frac{dV}{dt} = V_{\infty} - V
\end{equation} 

Note, the minus on the V is convenient for solving ODE.

Dividing both side by $\tau$:

\begin{equation}
\frac{dV}{dt} = \frac{V_{\infty} - V}{\tau}
\end{equation} 

\subsubsection{SOLVE ODE}

Separate the variables ($V$ and $t$), multiplying both sides by $dt$ and dividing both by $V_{\infty} - V$:

\begin{equation}
\frac{1}{V_{\infty} - V} dV =  \frac{1}{\tau} dt
\end{equation}

Define:

\begin{equation}
z \equiv V_{\infty} - V
\end{equation}

So WHERE DOES MINUS SIGN COME FROM:

\begin{equation}
\frac{1}{z} dz =  - \frac{1}{\tau} dt
\end{equation}

Then integrate both sides:

\begin{equation}
\int_{z(0)}{z(t)} \frac{1}{z} dz =   \int_ \frac{1}{\tau} dt
\end{equation}

Take anti-derivatives:

\begin{equation}
\ln z(t) - \ln z(0)  =  - \frac{t}{\tau}
\end{equation}

\begin{equation}
\ln z(t) - \ln z(0)  =  - \frac{t}{\tau}
\end{equation}

Difference of the logs is the log of the quotient:

\begin{equation}
\ln ( \frac{z(t)}{z(0)} )  =  - \frac{t}{\tau}
\end{equation}

Take the exponent of both sides:

\begin{equation}
\frac{z(t)}{z(0)} =  e^{\frac{t}{\tau}}
\end{equation}

\begin{equation}
z(t) = z(0) e^{\frac{t}{\tau}}
\end{equation}

The originals for $z$ are:

\begin{equation}
z(t) = V_{\infty} - V(t)
\end{equation}

\begin{equation}
z(t) = V_{\infty} - V(0)
\end{equation}

Restoring these in the equation gives us:

\begin{equation}
V_{\infty} - V(t) = ( V_{\infty} - V(0) ) e^{\frac{t}{\tau}}
\end{equation}

Subtract $V_{\infty}$ from both sides: 

\begin{equation}
- V(t) = - V_{\infty} + ( V_{\infty} - V(0) ) e^{\frac{t}{\tau}}
\end{equation}

Reverse the sign on both sides:

\begin{equation}
V(t) =  V_{\infty} + ( V(0) - V_{\infty} ) e^{\frac{t}{\tau}}
\end{equation}












Modeling the evolution of voltage over time.

Let us inject a current $I(t)$ into a RC circuit with a battery of voltage $V_{rest}$. The total current can be divided into a resistor component and a capacitor component (Why?):

\begin{equation}
I(t) = I_C (t) + I_R (t) 
\end{equation}

Recall that $C = \frac{Q}{V} \rightarrow Q = VC$ and $I = \frac{d}{dt} Q$ (charge per unit of time) so:

\begin{equation}
I_C =  \frac{d}{dt} VC = \frac{dV}{dt} C
\end{equation}

Recall that $V = IR$ or, for this case, $V_R = I_R R$. Given the battery, with voltage $V_{rest}$, the total voltage in the circuit $V = V_{rest} + V_R$, so $V_R = V - V_{rest}$ and:

\begin{equation}
I_R  =  \frac{1}{R} (V - V_{rest})
\end{equation}

Summing the last two equations:

\begin{equation}
I(t) = I_C (t) + I_R (t) = \frac{1}{R} (V - V_{rest}) +  \frac{dV}{dt} C
\end{equation}

Multiply both sides by R:

\begin{equation}
R I(t) =  (V - V_{rest}) + \frac{dV}{dt} RC
\end{equation}

Isolate last term:

\begin{equation}
\frac{dV}{dt} RC =  - (V - V_{rest}) + R I(t)
\end{equation}

\subsection{Linear Differential Equation}

Let's look at the units. The right hand side is in units of volts (recall that $V = IR$). The left hand side has one component $\frac{dV}{dt}$ with volts divided by units of time, so the other component must be a unit of time. We'll call this "time constant" $\tau$: $RC = \tau$, so substituting $\tau$ for $RC$:

\begin{equation}
\tau \frac{dV}{dt} =  \tau \frac{d}{dt} V = - (V - V_{rest}) + R I(t)
\end{equation}

Note, if we set a $V_1 = V - V_{rest}$ then $\frac{d}{dT}(V - V_{rest}) = \frac{d}{dT}V - \frac{d}{dT} V_{rest}) = \frac{d}{dT}V$ ($V_{rest}$ is a constant). And we cna say:

\begin{equation}
\tau \frac{dV_1}{dt} = - V_1 + R I(t)
\end{equation}  

Consider the solution for $I = 0$:

\begin{equation}
\tau \frac{dV_1}{dt} = - V_1
\end{equation}  

So:

\begin{equation}
V = e^{\frac{-t}{\tau}} \text{ that is, exponential decay over time, back to $V_{rest}$}
\end{equation}  

\end{document}